\documentclass[12pt,leqno,a4paper,oneside]{amsart}
\usepackage[utf8]{inputenc}
\usepackage{tikz}
\usetikzlibrary{automata,positioning}
\usepackage{booktabs}
\usepackage{amssymb}
\usepackage{hyperref}
\usepackage{graphicx}
\usepackage{textcomp}
\usepackage{bm}
\usepackage{caption}
\usepackage{float}
\usepackage{hhline}
\usepackage[parfill]{parskip}

\usepackage{listings}
\lstset{language=Python}
\usepackage{color}
\definecolor{mygreen}{rgb}{0,0.6,0}
\definecolor{mygray}{rgb}{0.5,0.5,0.5}
\definecolor{mymauve}{rgb}{0.58,0,0.82}

\lstset{ %
  backgroundcolor=\color{white},   % choose the background color; you must add \usepackage{color} or \usepackage{xcolor}
  basicstyle=\footnotesize,        % the size of the fonts that are used for the code
  breakatwhitespace=false,         % sets if automatic breaks should only happen at whitespace
  breaklines=true,                 % sets automatic line breaking
  captionpos=b,                    % sets the caption-position to bottom
  commentstyle=\color{mygreen},    % comment style
  deletekeywords={...},            % if you want to delete keywords from the given language
  escapeinside={\%*}{*)},          % if you want to add LaTeX within your code
  extendedchars=true,              % lets you use non-ASCII characters; for 8-bits encodings only, does not work with UTF-8
  frame=single,	                   % adds a frame around the code
  keepspaces=true,                 % keeps spaces in text, useful for keeping indentation of code (possibly needs columns=flexible)
  keywordstyle=\color{blue},       % keyword style
  language=Octave,                 % the language of the code
  otherkeywords={*,...},           % if you want to add more keywords to the set
  numbers=left,                    % where to put the line-numbers; possible values are (none, left, right)
  numbersep=5pt,                   % how far the line-numbers are from the code
  numberstyle=\tiny\color{mygray}, % the style that is used for the line-numbers
  rulecolor=\color{black},         % if not set, the frame-color may be changed on line-breaks within not-black text (e.g. comments (green here))
  showspaces=false,                % show spaces everywhere adding particular underscores; it overrides 'showstringspaces'
  showstringspaces=false,          % underline spaces within strings only
  showtabs=false,                  % show tabs within strings adding particular underscores
  stepnumber=2,                    % the step between two line-numbers. If it's 1, each line will be numbered
  stringstyle=\color{mymauve},     % string literal style
  tabsize=2,	                   % sets default tabsize to 2 spaces
  title=\lstname                   % show the filename of files included with \lstinputlisting; also try caption instead of title
}

\newtheorem{proclaim}{Lause}[section]
\newtheorem{lemma}[proclaim]{Lemma}
\newtheorem{corollary}[proclaim]{Seuraus}
\theoremstyle{definition}
\newtheorem{definition}[proclaim]{Määritelmä}
\newtheorem{example}[proclaim]{Esimerkki}
\newtheorem{exercise}{T.}
\theoremstyle{remark}
\newtheorem{remark}[proclaim]{Huomautus}
\numberwithin{equation}{section}
\renewcommand{\figurename}{Kuva}
\pagenumbering{gobble}

\begin{document}

\title{MAB8: Vastaukset}
\maketitle

\textsc{Vastauksista on spoilereiden välttämiseksi ilmoitettu vain desimaalimuoto. Se ei tarkoita, että desimaalimuoto olisi suositeltavin tapa
ilmoittaa vastaus. Huomaa myös pyöristysten vaikutus. Mukana on välillä enemmän desimaaleja kuin tarpeeksi, että vastauksista näkisi
paremmin, onko tehtävä laskettu oikein.}

\exercise{a) $24$  b) $120$ c) $720$.}
\label{1-1}

\exercise{a) 3 628 800 b) $2.4\cdot 10^18$.}

\exercise{$1.3\cdot 10^12$.}
\label{1-2}


\exercise{a) 241 920 b) 241 920}
\label{1-3}

\exercise{3 125}
\label{1-4}

\exercise{10000}
\label{1-5}


\exercise{60}

\exercise{a) 594 823 321 b) 342 014 400}


\exercise{a) 10 b) 10 Kysy lukujen merkitystä, jos et ole varma.}

\exercise{a) 210 b) 56.}

\exercise{715}

\exercise{5 148}

\exercise{21}

\exercise{15 380 937}
\label{lottolkm}

\exercise{0.0769}

\exercise{a) 0.000 000 065 b) 0.000 000 650}

\exercise{a) 0.001 189 b) 0.001231}

\exercise{a) 0.25 b) 0.308 c) 0.019}

\exercise{0.46}

\exercise{0.41, 0.65}

\exercise{0.004 115}

\exercise{0.345}

\exercise{a) 0.030 b) 0.970
c) $2.08\cdot 10^{-6}$}

\exercise{a) Ei $P\approx 0.167$ b) On $P\approx 0.538$ c) Pienenvät.}


\begin{exercise}a) 0.009 653
b) 0.048
\end{exercise}

\begin{exercise}
0.999 023
\end{exercise}

\begin{exercise}
0.989 221
\end{exercise}

\begin{exercise}
%lotto
0.781
\end{exercise}

\begin{exercise}
 0. 492 917
\end{exercise}

\begin{exercise}
0.602
\end{exercise}


\begin{exercise}
0.255
\end{exercise}


\begin{exercise}
0.556
\end{exercise}

\begin{exercise}
 a) 0.03215 b) 0.03549
\end{exercise}

\begin{exercise}
0.007497
\end{exercise}

\begin{exercise}
 a) 0.1675241321
 b) 0.7655
 c) 0.1674989533, eroaa 0.000025178
\end{exercise}

\begin{exercise}a) 0.0821 b) $4.5439\cdot 10^{-21}$ c) $1.6499\cdot 10^{-16}$  
\end{exercise}



\section{LibreOffice Calcista}

\begin{exercise}
  Libreoffice calc opas, libreoffice calc tutorial, libreoffice calc video tutorial, getting started with libreoffice calc,
  libreoffice calc aloitus, libreoffice calc perusteet
\end{exercise}

\begin{exercise}
Kysy, jos jokin mietityttää.
\end{exercise}

\begin{exercise}
 c) 14,5 
 h) 3,25
\end{exercise}

\begin{exercise}
 c) todennäköisin 5, epätodennäköisin 0.
 d) Luku 1 kertoo, että tehtävä on luultavasti laskettu oikein. Muut luvut kertovat, että se on laskettu väärin.
\end{exercise}


\begin{exercise}
\label{kannatustehtava}
 a) 0.743484
 b) 0.644168
 c) Nettigalluppeihin vastaavat eivät ole satunnainen otos koko väestöstä, ja voivat olla paljonkin vääristyneitä lukijakunnan ja 
    klikkailuinnon mukaan. Lisäksi samat henkilöt ovat voineet vastata galluppiin monta kertaa.
\end{exercise}




\begin{exercise}
%T42
 Kysy opettajalta/kaverilta, menikö oikein.
\end{exercise}

\begin{exercise}
a) 0.9453125 b) 0.989257813
\end{exercise}

\begin{exercise}
a) 0.000075083 b) 0.006159
\end{exercise}

\begin{exercise}
% T45
 a) 0.103639
\end{exercise}

\begin{exercise}
% T46
a) 267,5 ja 312,5
b) 0.958 635
\end{exercise}

\begin{exercise}
 a) 7.143 b) 6 c) 7 ja 8
\end{exercise}

\begin{exercise}
 0.707
\end{exercise}

\begin{exercise}
% T49
 A: $\bar{x}=1.5195$, $\sigma=0.0837630587$ B: $\bar{x}=2.5195$, $\sigma=0.0837630587$ C: $\bar{x}=3.5195$, $\sigma=0.0837630587$
\end{exercise}

\begin{exercise}
 a) $\bar{x}=3,418$, $\sigma = 0.377$ b) $\bar{x}=3,26$, $\sigma = 0.465$
\end{exercise}

\begin{exercise}
 0.477
\end{exercise}

\begin{exercise}
 a) 0.818 b) 148
\end{exercise}

\begin{exercise}
 a) 0.04799 b) 0.000429 c) 0.132
\end{exercise}

\begin{exercise}
 a) 0.000159 b) 0.336
\end{exercise}

\begin{exercise}
 a) 0.4308 b) 0.000108
\end{exercise}


\begin{exercise}
  a) 1012.36 g
  b) 0.632
  c) 990.69 g
\end{exercise}

\begin{exercise}
  215.4 cm
\end{exercise}




\begin{exercise}
 a) kyllä
 b) ei
 c) kyllä
 d) ei.
\end{exercise}



\begin{exercise}
 a) 141.1818 mmHg, 172.84 cm
 b) tiedostossa on vain \emph{aikuisten} verenpaineita ja pituuksia, joten se ei ole otos \emph{kaikista} suomalaisista.
\end{exercise}

\begin{exercise} 
 a) 68.9\%
 b) 67.7\%
 c) 12.9\%
\end{exercise}

\begin{exercise}
$\mu= 169.1$ cm, ennuste keskihajonnalle on otoskeskihajonta $\sigma_{n-1} = 9.85$ cm
\end{exercise}

\begin{exercise}
$\mu=1.326$ cm, $\sigma_{n-1} = 0.198$ cm.
\end{exercise}



\begin{exercise}
 $[136.70\text{ mmHg }, 150.98\text{ mmHg }]$  
\end{exercise}


\begin{exercise}
$[131.85\text{ mmHg }, 155.83\text{ mmHg }]$ 
\end{exercise}

\begin{exercise}
 a) $[172.7\text{ cm }, 179.3\text{ cm }]$
 b) $[171.7\text{ cm }, 180.3\text{ cm }]$
\end{exercise}




\begin{exercise}
a) $[68.4\%, 74.0\% ]$ 
b) $[67.5\%, 74.9\% ]$ 
\end{exercise}

\begin{exercise}
 Haatainen: $[0.8\%, 3.6\% ]$, Vanhanen $[2.4\%, 6.2\% ]$
\end{exercise}


\begin{exercise} (Tilanne 14.5.2018)
 a) Perussuomalaiset\\
 b) $[15.2\%, 54.2\% ]$
 c) Eivät: vauva-lehden lukijat eivät ole otos koko populaatiosta. (Lisäksi vastaajia on hyvin vähän, jopa niin vähän, ettei 
 luottamusväliä voisi laskea opetetulla tavalla, mutta se ei ole tämän kurssin asia).
\end{exercise}

\begin{exercise}
 a) 0.0387595 b) 0.136798
\end{exercise}


\begin{exercise}
%T70
a) 120 b) 2 
\end{exercise}


\begin{exercise}
 %T71
 0.175
\end{exercise}

\begin{exercise}
%T72
 0-3 oikein, 35/120, 63/120, 21/120 ja 1/120. Todennäköisyys yhteensä 1.
\end{exercise}

\begin{exercise}
 %T73
 0.940765
\end{exercise}

\begin{exercise}
 0.056895
\end{exercise}

\begin{exercise}
0.588235, tn epäonnistua ainakin toisessa 0.32 
\end{exercise}

\begin{exercise}
 %T76
 0.28
\end{exercise}

\begin{exercise}
 %T77
 25.2\% alle 200 g, 58.9\% 200-210 g.
\end{exercise}

\begin{exercise}
 %T78
 0.7358395, 0.641514
\end{exercise}

\begin{exercise}
 %T79
 0.022750
\end{exercise}

\begin{exercise}
 %T80
 1.2159$^{\circ}$C
\end{exercise}


\begin{exercise}
 %R81
 100
\end{exercise}

\begin{exercise}
 %T82
On. Odotusarvo on 0.50 euroa.
\end{exercise}

\begin{exercise}
 %T83
 -7 778 pelirahaa.
\end{exercise}










\end{document}

