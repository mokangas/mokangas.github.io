\documentclass[12pt,leqno,a4paper,oneside]{amsart}
\usepackage[utf8]{inputenc}
\usepackage{tikz}
\usetikzlibrary{automata,positioning}
\usepackage{booktabs}
\usepackage{amssymb}
\usepackage{hyperref}
\usepackage{graphicx}
\usepackage{textcomp}
\usepackage{bm}
\usepackage{caption}
\usepackage{float}
\usepackage{hhline}
\usepackage[parfill]{parskip}

\usepackage{listings}
\lstset{language=Python}
\usepackage{color}
\definecolor{mygreen}{rgb}{0,0.6,0}
\definecolor{mygray}{rgb}{0.5,0.5,0.5}
\definecolor{mymauve}{rgb}{0.58,0,0.82}

\lstset{ %
  backgroundcolor=\color{white},   % choose the background color; you must add \usepackage{color} or \usepackage{xcolor}
  basicstyle=\footnotesize,        % the size of the fonts that are used for the code
  breakatwhitespace=false,         % sets if automatic breaks should only happen at whitespace
  breaklines=true,                 % sets automatic line breaking
  captionpos=b,                    % sets the caption-position to bottom
  commentstyle=\color{mygreen},    % comment style
  deletekeywords={...},            % if you want to delete keywords from the given language
  escapeinside={\%*}{*)},          % if you want to add LaTeX within your code
  extendedchars=true,              % lets you use non-ASCII characters; for 8-bits encodings only, does not work with UTF-8
  frame=single,	                   % adds a frame around the code
  keepspaces=true,                 % keeps spaces in text, useful for keeping indentation of code (possibly needs columns=flexible)
  keywordstyle=\color{blue},       % keyword style
  language=Octave,                 % the language of the code
  otherkeywords={*,...},           % if you want to add more keywords to the set
  numbers=left,                    % where to put the line-numbers; possible values are (none, left, right)
  numbersep=5pt,                   % how far the line-numbers are from the code
  numberstyle=\tiny\color{mygray}, % the style that is used for the line-numbers
  rulecolor=\color{black},         % if not set, the frame-color may be changed on line-breaks within not-black text (e.g. comments (green here))
  showspaces=false,                % show spaces everywhere adding particular underscores; it overrides 'showstringspaces'
  showstringspaces=false,          % underline spaces within strings only
  showtabs=false,                  % show tabs within strings adding particular underscores
  stepnumber=2,                    % the step between two line-numbers. If it's 1, each line will be numbered
  stringstyle=\color{mymauve},     % string literal style
  tabsize=2,	                   % sets default tabsize to 2 spaces
  title=\lstname                   % show the filename of files included with \lstinputlisting; also try caption instead of title
}

\newtheorem{proclaim}{Lause}[section]
\newtheorem{lemma}[proclaim]{Lemma}
\newtheorem{corollary}[proclaim]{Seuraus}
\theoremstyle{definition}
\newtheorem{definition}[proclaim]{Määritelmä}
\newtheorem{example}[proclaim]{Esimerkki}
\newtheorem{exercise}{Tehtävä}
\theoremstyle{remark}
\newtheorem{remark}[proclaim]{Huomautus}
\numberwithin{equation}{section}
\renewcommand{\figurename}{Kuva}


\begin{document}

\title{MAB8: Tehtävät}
\maketitle

\section{Kombinatoriikkaa}


\exercise{Laske kynällä ja paperilla a) $4!$  b) $5!$ c) $6!$.}
\label{1-1}

\exercise{Laske laskimella tai LibreOfficella a) $10!$ b) $20!$.}

\exercise{Kuinka monella eri tavalla kirjaimet F,G,...,S,T voidaan järjestää? Vinkki: laske kirjainten lukumäärä sormilla.}
\label{1-2}


\exercise{Kuinka monella tavalla kirjaimet E,F,G,...,N,O voidaan laittaa järjestykseen niin, että a) vokaalit ovat ennen konsonantteja
b) konsonantit ovat ennen vokaaleja?}
\label{1-3}

\exercise{Montako viisikirjaimista sanaa voidaan muodostaa kirjaimista A,B,C,D,E? Sanojen ei tarvitse tarkoittaa mitään, eli esimerkiksi
DCCDC kelpaa sanaksi.}
\label{1-4}

\exercise{Montako nelinumeroista pin-koodia on olemassa?}
\label{1-5}


\exercise{Kuinka monta anagrammia sanasta SAAPAS voidaan muodostaa, jos sanojen ei tarvitse tarkoittaa mitään?}

\exercise{Pussissa on 29 palloa, joissa jokaisessa on eri kirjain. Niistä muodostetaan satunnainen sana nostamalla kuusi palloa yksi kerrallaan.
Kuinka monta erilaista sanaa voidaan saada, jos a) pallot laitetaan noston jälkeen takaisin pussiin, b) palloja ei laiteta noston jälkeen
takaisin pussiin?}


\exercise{Laske kynällä ja paperilla a) ${5\choose 2}$ b) ${5\choose 3}$. Mitä nämä luvut merkitsevät?}

\exercise{Laske koneella a) ${10\choose 4}$ b) ${8\choose 3}$.}

\exercise{Montako neljän kortin kättä voidaan nostaa korttipakan hertoista?}

\exercise{Kuinka monessa viiden kortin kädessä kaikki kortit ovat samaa maata?}

\exercise{Heitetään kahta noppaa (kuten esim. Monopolissa tai backgammonissa). Jos tulos on vaikkapa 2 ja 6, on samantekevää, kumpi
noppa on kakkonen ja kumpi kutonen. Montako erilaista tulosta voidaan saada? (Tätä varten kannattaa piirttää jonkinlainen taulukko silmä\-luvuista).}

\exercise{Lotossa arvotaan 39:stä numerosta 7 voittonumeroa. Kuinka monta erilaista lottoriviä on olemassa? (Unohdetaan lisänumerot toistaiseksi)}
\label{lottolkm}

\exercise{Pakasta nostetaan satunnainen kortti. Millä todennäköisyydellä se on ässä?}

\exercise{Millä todennäköisyydellä yksi rivi voittaa lotossa? (Käytä tehtävää \ref{lottolkm} apuna). b) Matti on pelannut kymmenen eri riviä
lottoa. Millä todennäköisyydellä hän saa täysosuman?}

\exercise{Annelilla on pussissa 29 palloa, joissa jokaisessa on eri kirjain. Hän arpoo kaksikirjaimisen sanan. Millä todennäköisyydellä sana on ``EU'',
jos a) hän palauttaa ensimmäisen pallon pussiin noston jälkeen b) ei palauta sitä?}

\subsection{Sanastoa ja merkintöjä}

\exercise{
Sovitaan, että 
$$A=\text{kortti on pata, ja}$$
$$B=\text{kortti on nelonen.}$$
Laske todennäköisyydet
\begin{itemize}
 \item[a)]
 $P(A)$
 \item[b)]
 $P(A\text{ tai } B)$
 \item[c)]
 $P(A\text{ ja } B)$?
\end{itemize}
}


\subsection{Jotkut vaihtoehdot ovat todennäköisempiä kuin toiset}


\exercise{Jos kolikonheitossa kruunan todennäköisyys on 0.54, mikä on klaavan todennäköisyys?}

\exercise{Pelurin nopan todennäköisyydet silmäluvuille 1-5 ovat seuraavan taulukon mukaisia:


 \begin{center}
 \begin{tabular}{l|l}
  Silmäluku & P\\
  \hline
  1 & 0.15\\
  2 & 0.03 \\
  3 & 0.11 \\
  4 & 0.21 \\
  5 & 0.09 \\
 \end{tabular}
 \end{center}

 Millä todennäköisyydellä hän heittää kutosen? Millä todennäköisyydellä heiton tulos on parillinen?
}

\exercise{Heitetään viittä noppaa. Millä todennäköisyydellä ei saada vitosta pienempiä silmälukuja?}

\exercise{Korttipakasta nostetaan kuusi korttia. Millä toden\-näköisyydellä kaikissa on eri numero/kirjain?}

\exercise{Uhkapeluri seuraa lottoarvontaa, jossa nostetaan 39:stä numerosta 7 voittonumeroa. Hän on saanut kuusi ensimmäistä numeroa oikein.
a) Mikä on todennäköisyys, että seitsemäs numero on oikein? b) Mikä on todennäköisyys, että seitsemäs numero on väärin?
c) Mikä on todennäköisyys ennen arvontaa, että kuusi ensimmäistä arvottua numeroa tulevat osumaan pelurin riviin, mutta seitsemäs ei?}

\exercise{Opettaja lyö vetoa siitä, onko luokassa oppilaita, joilla on sama syntymäpäivä (syntymävuosi voi olla eri). 
Oleta, että jokaisessa vuodessa on 365 päivää.
Onko veto opettajan kannalta järkevä, jos oppilaita on a) 12 b) 24? c) Jos karkausvuodet huomioidaan, pieneneekö vai suureneeko
opettajan voitonmahdollisuudet?}


\begin{exercise}
Helsingissä oli 628 208 asukasta 1.1.2017. Heistä 105 240 oli alaikäisiä ja 36 197:n äidinkieli oli ruotsi. Oleta, että äidinkielellä
ja henkilön iällä ei ole riippuvuutta (ts. ruotsinkielisten osuus kaikissa ikäluokissa on yhtä suuri). 
Millä todennäköisyydellä sattumanvaraisesti valittu helsinkiläinen on a) ruotsinkielinen ja alaikäinen
b) ruotsinkielinen ja täysikäinen?\footnote{\url{https://www.hel.fi/hel2/tietokeskus/julkaisut/pdf/17_06_28_Tilastoja_1_Maki_Vuori.pdf}} 
\end{exercise}

\begin{exercise}
 Peluri heittää kolikko kymmenen kertaa. Mikä on toden\-näköisyys, että hän saa ainakin yhden kruunan?
\end{exercise}

\begin{exercise}
 Pelurin kolikko laskeutuu klaava ylöspäin todennäköisyydellä $0.53$. Millä toden\-näköisyydellä hän saa kuudella heitolla ainakin yhden klaavan?
\end{exercise}


\begin{exercise}
 Lotossa arvotaan 39:stä numerosta 7, ja sitä pelataan 7 numeron rivillä. Mikä on todennäköisyys, että rivissä on ainakin yksi numero oikein?
\end{exercise}

\begin{exercise}
 Korttipakasta nostetaan viiden kortin käsi. Millä toden\-näköisyydellä ainakin kahdessa kortissa on sama numero/kirjain?
\end{exercise}

\begin{exercise}
 Korttipakasta nostetaan kolme korttia. Millä toden\-näköisyydellä ainakin kaksi niistä on samaa maata?
\end{exercise}


\begin{exercise}
 Mikä on todennäköisyys sille, että toinen pakasta nostettu kortti on pata ehdolla, että ensimmäinen oli hertta?
\end{exercise}


\begin{exercise}
 Laske todennäköisyys sille, että kahden nopan heitolla saadaan ainakin yksi kutonen ehdolla, että molemmat silmäluvut ovat parillisia.
\end{exercise}



\section{Toistokoe ja binomitodennäköisyydet}

\begin{exercise}
 Heitetään viittä noppaa. Millä todennäköisyydellä saadaan a) tasan kolme kutosta b) Vähintään kolme kutosta?
\end{exercise}

\begin{exercise}
 Henkilö A matkustaa pummilla saman reitin 40 kertaa kuussa. Sakkojen saamisen todennäköisyys jokaisella matkalla on 0.01. Millä todennäköisyydellä
 A saa useammat kuin kahdet sakot kuussa?
\end{exercise}


\begin{exercise}
 Helsingissä on 628 208 asukasta, ja heistä 105 240 on alaikäisiä.\\
 a) Jos valitaan satunnainen helsinkiläinen, millä todennäköisyydellä hän on alaikäinen?\\
 b) Jos valitaan 20 satunnaista helsinkiläistä, millä todennäköisyydellä joukossa on enintään neljä alaikäistä?
 Oleta, että valinnan todennäköisyys on jokaisella valinnalla vakio (se, jonka laskit 
 a-kohdassa). Nspirellä saat laskettua binomitodennäköisyyden suoraan komennolla \texttt{binomPdf(n,p,k)}, missä $n$ on kokeiden lukumäärä,
 $p$ onnistumisen todennäköisyys ja $k$ onnistuneiden kokeiden todennäköisyys. LibreOfficen ohje on tiivistelmässä ja myöhemmin monisteessa, laatikossa ennen 
 tehtävää \ref{lolaatikko}.\footnote{Jos tehtävässä kysytään vain yhtä todennäköisyyttä, käytä mieluummin binomitodennäköisyyden kaavaa.
 Jos laskeminen olisi muuten vaivalloista, voit käyttää \texttt{binomPdf}:ää tai vastaavaa.}\\
 c) Tarkastellaan vielä b-kohdan oletuksen vakiotodennäköisyydestä jär\-ke\-vyyt\-tä. Jos 19 ensimmäistä ryhmään valittua helsinkiläistä oli alaikäisiä,
 mikä on \emph{oikea} todennäköisyys sille, että myös kahdeskymmenes on? Kuinka paljon se eroaa a-kohdan todennäköisyydestä?
\end{exercise}

\begin{exercise}
 Oletetaan, että Niinistöä kannattaa presidentiksi 62.6\% äänioikeutetuista.\footnote{Niinistö voitti tällä kannatuksella vaalit 2018, mutta
 äänestysprosentti oli vain 66,7\%, joten tehtävän oletus on virheellinen. Se tekee laskemisesta kuitenkin paljon mukavampaa.}
 Laiskassa vaaligallupissa valitaan satunnaisesti sata äänioikeutettua.
 Millä todennäköisyydellä Niinistön kannatus on gallupissa a) tasan 63\% b) tasan 100\% c) vähintään 97 \% ? 
\end{exercise}



\section{LibreOffice Calcista}

\begin{exercise}
 Mitkä ovat hyviä hakusanoja LibreOffice Calcin käytön aloittamiseen? Kokeile niitä hakukoneeseen ja laita bookmarkkeihin parhaat.
\end{exercise}

\begin{exercise}
\label{loex1}
 Katso tehtävän asiat läpi, jos ne ovat sinulle jo entuudestaan tuttuja, älä turhaan näe vaivaa, vaan hyppää seuraavaan tehtävään.\\
 1)Tee seuraavanlainen tiedosto mielikuvitusoppilaiden pisteistä:
 
 \includegraphics[width=10cm]{localc1.png}
 
 Syötettyäsi numeron pystyt siirtymään alaspäin enterillä ja oikealla sarkaimella (Caps Lockin yläpuolella). 
 
 2) Klikkaa ruutua G2 ja maalaa vasemmalle ruutuun B2 asti:
 
 \includegraphics[width=10cm]{localc2.png}
 
 3) Klikkaa ruutujen yläpuolella olevaa isoa sigmaa: $\Sigma$ . Mitä LibreOffice laski ruutuun G2?
 
 \includegraphics[width=10cm]{localc3.png}
 
 4) Klikkaa ruutua G2. Ruudun oikeassa alakulmassa on pieni musta neliö. Tartu siihen hiirellä ja vedä neljä riviä alaspäin:
 
 \includegraphics[width=10cm]{localc4.png}
 
 5. Tallenna tiedosto .ods- ja .csv-muodoissa: Kun olet klikannut ``save as'', aukeaa ikkuna, jossa voit valita tallennuspaikan, tiedoston nimen ja muodon.
 Tiedoston muoto on todennäköisesti ikkunassa alhaalla save-napin yläpuolella, voit valita siitä ``text csv''. Jos siinä kohdassa lukee ``all formats''.
 Voit päättää tiedoston muodon myös kirjoittamalla nimen päätteeksi ``.ods'' tai ``.csv'', esimerkiksi \texttt{mab8\_tehtava37.csv}.  
 Tätä tiedostoa tullaan käyttämään seuraavissa tehtävissä.
\end{exercise}


\begin{exercise}
 Katso tehtävän asiat läpi, jos ne ovat sinulle jo entuudestaan tuttuja, älä turhaan näe vaivaa, vaan hyppää seuraavaan tehtävään.\\
 Avaa tehtävässä \ref{loex1} tehty .ods-muotoinen tiedosto.\\
 a) Laske ruutuun G6 oppilaiden saamien kokonaispisteiden summa.\\
 b) Laske ruutuun A6 oppilaiden määrä: \texttt{=COUNTA(A2:A5)}.\\
 c) Laske ruutuun H6 oppilaiden kokonaispistemäärän keskiarvo: \texttt{=[jokin ruutu]/[jokin ruutu]}.\\
 d) Laske ruutuun E6 kaikkien oppilaiden kaikkien tehtävien pistemäärä suoraan tehtäväruuduista: \texttt{=SUM(B2:E5)}.\\
 e) Klikkaa ruutua E6, ota pienestä neliöstä sen vasemmassa alakulmassa kiinni ja vedä alaspäin. Mitä alapuolella oleviin ruutuihin tulee,
 ja miksi? \\
 f) Poista e-kohdan laskut ja laske ruutuun B6 summa kaikkien oppilaiden tehtävästä 1 saamista pisteistä. \\
 g) Ota ruudusta B6 kiinni ja maalaa oikealle niin, että saat jokaiseen tehtävään yhteispistemäärän.\\
 h) Laske ruutuun B7 ykköstehtävstä saatujen pisteiden keskiarvo tällä tavalla \texttt{=[jokin ruutu]/[jokin ruutu]} .\\
 i) Ota ruudusta B7 kiinni ja maalaa oikealle, että saisit jokaisesta tehtävästä keskimääräiset pisteet. Miksi tämä ei toimi?\\
 j) Muuta ruudun B7 sisällöksi \texttt{=B6/\$A\$6}, ota ruudusta kiinni ja maalaa oikealle. Katso ruutujen C7-E7 sisältöä.
\end{exercise}

\begin{exercise}
\label{lolaatikko}
 Avaa uusi dokumentti LibreOffice Calcissa ja kopioi siihen vasemmanpuoleinen kuva. Maalaa sitten ruudut A2 ja A3, tartu A3:n oikeassa alakulmassa
 olevaan pieneen neliöön ja vedä alaspäin kunnes ruuduissa on numerot 0-10.
 \begin{figure}[H]
 \begin{center} 
 \begin{minipage}{.45\textwidth}
 \includegraphics[width=5cm]{localc5}
 \end{minipage}
 \begin{minipage}{.45\textwidth}
 \includegraphics[width=5cm]{localc6}
 \end{minipage}
 \end{center}
 \end{figure}
 
 a) Ajatellaan koetta, jossa kolikkoa heitetään kymmenen kertaa. Kruunan todennäköisyys on 0.52. Laske ruutuun B2 binomitodennäköisyys nollalle
    kruunalle. Älä kuitenkaan käytä funktion ensimmäisenä argumenttina lukua 0, vaan viittaa ruutuun A2 ilman dollareita.\\
    
 b) Ota ruudusta kiinni ja raahaa binomitodennäköisyydet ruutuihin B3-B12.
 
 c) Montako kruunaa on todennäköisin tulos? Entä epätodennäköisin?
 
 d) Laske binomitodennäköisyyksien summa ruutuun B13. Mitä saatu luku kertoo tehtävän ratkaisusta?
\end{exercise}


\begin{exercise}
\label{kannatustehtava}
 Oletetaan, että 62,6\% kaikista äänioikeutetuista äänestäisi Sauli Niinistöä, jos vaalit järjestettäisiin nyt.\\
 a) Laiska gallupin tekijä kysyy sadalta satunnaiselta äänioikeutetulta, ketä he äänestäisivät. Millä todennäköisyydellä hän saa Niinistön kannatukseksi
 58\% -68\% (eli n. viiden prosenttiyksikön sisälle todellisesta kannatuksesta)?\\
 b) Vielä laiskempi gallupin tekijä haastattelee vain kahtakymmentä sa\-tun\-nais\-ta äänioikeutettua. Millä todennäköisyydellä hän saa Niinistön kannatukseksi
 55\% - 70\%? \\
 c) Iltapäivälehti kysyy nettisivuillaan lukijoilta, ketä he äänestäisivät presidentiksi. Tulosten mukaan Niinistö ei pääsisi edes toiselle kierrokselle.
 Onko tämä vain huonoa onnea, vai kenties salaliitto?
\end{exercise}


\begin{exercise}
 Alla on painotetun nopan todennäköisyydet. Laske oikeanpuoleiseen sarakkeeseen kertymäfunktio, eli todennäköisyydet, että
 silmäluku on enintään $n.$\\
 \begin{center}
\begin{tabular}{l|l|l}
Tulos &TN &Kertymäfunktio\\
\hline
1 &0.2&\\
2 &0.1&\\
3 &0.1&\\
4 &0.3&\\
5 &0.1&\\
5 &0.2&
\end{tabular}
\end{center}
\end{exercise}

\begin{exercise}
 Kolikko on reilu, eli kruuna ja klaava ovat yhtä todennäköiset. Sitä heitetään kymmenen kertaa. Laske todennäköisyys
 saada a) enintään 7 kruunaa b) vähintään 2 kruunaa. 
\end{exercise}

\begin{exercise}
Kolikonheitosta saadaan kruuna todennäköisyydellä 0.46. Millä todennäköisyydellä tuhannella heitolla saadaan
a) enintään 400 kruunaa b) vähintään 400 kruunaa?
\end{exercise}

\begin{exercise}
 Tutkijat pudottavat voileivän kaksisataa kertaa pöydältä lattialle, ja se putoaa 112 kertaa voipuoli alaspäin
 (merkitään $n=112$).
 Kokeen merkittävyyttä tarkastellaan vertaamalla sitä todennäköisyyteen saada tällainen tai äärimmäisempi tulos, jos 
 eri päin putoaminen olisi yhtä todennäköistä. Käytännössä siis tutkijoiden pitää laskea todennäköisyys
 $$P(n \leqq 88) + P(112\leqq n).$$
 a) Laske tämä todennäköisyys\\
 (Lisätietoa: tätä todennäköisyyttä sanotaan kokeen \emph{p-arvoksi}, kun nollahypoteesi on, että molemmin päin 
 putoaminen on yhtä todennäköistä. Lukua $p=0.05$ pidetään yleensä tilastollisen merkittävyyden rajana. Lukua käytetään siksi, että ennen tietokoneaikaa
  nämä asiat laskettiin käsin ja taulukoiden avulla, ja yleensä taulkoidut arvot olivat 0.05, 0.01 ja 0.001. Tämä luku siis on
  jokseenkin mielivaltaisesti valittu).
\end{exercise}

\begin{exercise}
Helsingin sanomien 22.-24. 1. 2018 teettämään galluppiin\footnote{\url{https://www.hs.fi/politiikka/art-2000005538817.html}} 
vastasi 500 äänioikeutettua. Niinistö kannatus oli 58\% ja virhemarginaali 4,5 prosenttiyksikköä.\\
a) Kuinka montaa galluppiin vastannutta prosenttimäärät 53,5\% ja 62,5\% vastaavat? (Älä pyöristä vielä lukuja)\\
b) Oleta, että Niinistöä todellisuudessa kannatti 58\% äänioikeutetuista. Millä todennäköisyydellä hänen kannatus on 
500 henkilön gallupissa ilmoitettujen virherajojen sisälpuolella? (Että kannatus olisi virherajojen sisäpuolella, sen on oltava suurempi
kuin alaraja ja pienempi kuin yläraja. Valitse a-kohdan pyöristykset tämän perusteella)
\end{exercise}



\begin{exercise}
 Oppilaan matematiikan arvosanat kursseilta MAB1-MAB7 ovat
 $$6,\, 7,\, 8,\, 6,\, 8,\, 9,\, 6.$$
 a) Mikä on hänen matematiikan keskiarvo tällä hetkellä?\\
 b) Mikä arvosana hänen on saatava kurssista MAB8, että keskiarvo olisi tasan 7?\\
 c) Loppuarvosana on kurssien keskiarvo pyöristettynä lähimpään kokonaislukuun (tasan 0.5-loppuinen ylöspäin).
 Mitkä ovat oppilaan suurin ja pienin mahdollinen loppuarvosana MAB8:n jälkeen?
\end{exercise}


\begin{exercise}
Esimerkissä 5.1 oppilas B sai arvosanat 8, 8, 7, 9, 8, 8, 9, 7. Laske hänen arvosanojensa keskihajonta.
\end{exercise}


\begin{exercise}
 Osoitteessa \url{https://www.cs.helsinki.fi/u/mokangas/mab8/data1.csv} on kolme sarjaa mittaustuloksia (A, B ja C). 
 Lataa tiedosto ja laske jokaisesta sarjasta keskiarvo ja keskihajonta. 
 (Mittaussarjat ovat eri esineiden terminaalinopeuksia putoamisen aikana).
 
 Huomaa, että tiedosto ei välttämättä aukea koneellasi oikein. Siinä tapauksessa kannattaa kysyä apua. Tiedoston saa helposti näkymään 
 oikein, mutta ohjeita lukemalla se on vaikeaa. Tämä on myös osa tehtävää, sillä tosimaailmassa väärin aukeavat tiedostot ovat yleinen 
 hankaluus, jonka kanssa kannattaa opetella tulemaan toimeen.
\end{exercise}

\begin{exercise}
 Osoitteessa \url{https://www.cs.helsinki.fi/u/mokangas/mab8/iris.csv} on Kurjenmiekkojen lehtien pituuksia ja leveyksiä.
 Lataa tiedosto ja laske keskiarvo ja -hajonta a) setosa-lajin sepal\_widthille b) versicolorin petal\_lengthille. (Mitat ovat senttimetrejä)
\end{exercise}



\begin{exercise}
Kahvipaketteissa on keskimäärin 500 grammaa kahvia. Laitteiston epätarkkuuden vuoksi paketin massan keskihajonta on 3 grammaa.
Millä todennäköisyydellä satunnaisessa kahvipaketissa on 500-506 grammaa kahvia?
\end{exercise}

\begin{exercise}
 Ihmisten älykkyysosamäärän keskiarvo on 100 ja keskihajonta 15.\footnote{ÄO on periaatteessa määritelty näin.}\\
 a) Millä todennäköisyydellä satunnaisesti valitun henkilön ÄO on 85-130?\\
 b) Joskus ÄO:n keskihajonnaksi on valittu 24. Jos henkilön ÄO on 130 keskihajonnalla 15, mikä se olisi keskihajonnalla 24?
\end{exercise}


 
\begin{exercise}
 Kahvipaketin massa on normaalijakautunut. Sen keskiarvo on 500 grammaa ja keskihajonta 3 grammaa.\\
 a) Millä todennäköisyydellä satunnaisessa kahvipaketissa on alle 495 g kahvia?\\
 b) Millä todennäköisyydellä satunnaisessa kahvipaketissa on yli 510 g kahvia?\\
 c) Kuluttajan vaaka näyttää arvoa 500 g, jos paketin todellinen massa on 499.5-500.5 grammaa. Millä todennäköisyydellä näin käy?
\end{exercise}

\begin{exercise}
 Kurjenmiekan terälehden pituus on normaalijakautunut. Sen keskiarvo on 3.8 cm ja keskihajonta 0.5 cm. \\
 a) Millä todennäköisyydellä satunnaisen terälehden pituus on alle 2 cm?\\
 b) Millä todennäköisyydellä satunnaisen terälehden pituus on 4-5 cm?
\end{exercise}

\begin{exercise}
 Kurjenmiekan terälehden pituus on normaalijakautunut. Sen keskiarvo on 3.8 cm ja keskihajonta 0.5 cm. \\
 a) Anni kertää kymmenen satunnaista kurjenmiekan terälehteä. Millä todennäköisyydellä ne ovat kaikki alle 4.5 cm pitkiä?\\
 b) Benjamin kerää kaksikymmentä satunnaista terälehteä. Millä todennäköisyydellä hän saa ainakin yhden kuusisenttisen?
\end{exercise}

\begin{exercise}
  Maitotölkin massa on normaalijakautunut keskiarvolla 1000 grammaa ja keskihajonnalla 4 grammaa.\\
  a) Kuinka painava tölkin on oltava ollakseen painavimmassa promillessa kaikista tölkeistä?\\
  b) Millä todennäköisyydellä tällainen tölkki on tuhannen satunnaisen tölkin joukossa?\\
  c) Minkä painon tölkittäjä voi ilmoittaa vähimmäispainoksi, jos halutaan, että alle prosentti tölkeistä on sitä kevyempiä?
\end{exercise}

\begin{exercise}
  Ovien valmistaja haluaa, että alle miljoonasosa ihmisistä joutuu kumartumaan kulkiessaan heidän ovistaan. Jos ihmisten keskipituus on
  175 cm ja keskihajonta 8.5 cm, kuinka korkeita ovien pitää olla?
\end{exercise}


\section{Tilastotiedettä}




\begin{exercise}
 Kuuluuko \href{https://fi.wikipedia.org/wiki/Antti_Tuisku}{Antti Tuisku} populaatioon, kun tutkitaan\\
 a) puoluekannatusta eduskuntavaaleissa 2019\\
 b) ruotsalaisten keskipituutta\\
 c) Euroopan nisäkkäiden elinikää\\
 d) tilastotieteilijöiden työllisyysastetta.
\end{exercise}



\begin{exercise}
 Osoitteessa \url{https://www.cs.helsinki.fi/u/mokangas/mab8/vpjapituus.csv} on tiedosto, jossa on yläverenpaine ja pituus pieneltä otokselta
 aikuisia suomalaisia. \\
 a) Määritä tiedostosta uskottavin arvo suomalaisten aikuisten yläveren\-paineelle ja pituudelle.\\
 b) Miksi nämä eivät ole uskottavimpia arvoja kaikkien suomalaisten yläverenpaineelle ja pituudelle?
\end{exercise}

\begin{exercise}
 Osoitteessa \url{https://www.cs.helsinki.fi/u/mokangas/mab8/gallup.csv} on satunnaisesti generoitu galluptutkimus presidentinvaaleihin 
 2019.\footnote{Vastaukset on arvottu vaalien todellisen äänten perusteella koneellisesti.} Listassa 0 tarkoittaa, että vastaaja ei äänestäisi,
 ja muulloin numero on sen ehdokkaan numero, jota hän äänestäisi.\\ 
 a) Laske uskottavin arvo äänestysprosentille. Nollat on helppo laskea näin: \texttt{=COUNTIF(A1:A45, ``=0'')}\\
 b) Laske uskottavin arvo Niinistön (8) kannatukselle. Huomaa, että kannatus lasketaan prosenttina äänestäneistä, ei äänioikeutetuista.\\
 c) Laske uskottavin arvo Haaviston (3) kannatukselle.
\end{exercise}




\begin{exercise}
 Tutkija yrittää selvittää norjalaisten aikuisten keski\-pituutta. Hän valitsee satunnaisen otoksen ja saa mitat
 $$179, 174, 171, 169, 172, 148 \text{ ja } 171 \text{ cm.} $$
 Mikä on hänen ennusteensa norjalaisten aikuisten pituudelle ja sen keskihajonnalle?
\end{exercise}

\begin{exercise}
 Osoitteessa \url{https://www.cs.helsinki.fi/u/mokangas/mab8/iris.csv} on Kurjenmiekkojen mittoja. Laske uskottavin arvo versicolor-lajikkeen terälehden
 leveydelle (petal\_width) ja sen otoskeskihajonta.
\end{exercise}




\begin{exercise}
 Suomalaisten aikuisten yläverenpainetta tutkitaan 50 henkilön otoksella, jossa keskiarvo yläpaineelle on 143.84 mmHg ja keskihajonta 25.77 mmHg.
 Mikä on 95\%:n luottamusväli tutkimuksen mukaan?
\end{exercise}


\begin{exercise}
 Etsi MAOLin taulukoista luottamusvälit ja laske edelliseen tehtävään 99.9\%:n luottamusväli.
\end{exercise}

\begin{exercise}
 Osoitteessa \url{https://www.cs.helsinki.fi/u/mokangas/mab8/pituuksia.csv} on otos amerikkalaisten aikuisten pituuksia. Laske sen perusteella\\
 a) 95 \%:n luottamusväli amerikkalaisten aikuisten keskipituudelle\\
 b) 99\%:n luottamusväli amerikkalaisten aikuisten keskipituudelle.
\end{exercise}




\begin{exercise}
 Vaaligallupissa haastateltiin tuhatta satunnaisesti valittua äänoikeutettua. Heistä 712 sanoi aikovansa äänestää.
 Mikä on gallupin perusteella 95\% luottamusväli
 vaalien äänestysprosentille?
\end{exercise}

\begin{exercise}
 Osoitteessa \url{https://www.cs.helsinki.fi/u/mokangas/mab8/gallup2.csv} on koneellisesti arvottu vaaligallup presidentinvaaleista 2018.
 Luku 0 tarkoittaa hylättyä ääntä ja muut numerot ehdokkaiden numeroita. Laske 95\%:n luottamusväli Haataisen (6) ja Vanhasen (4) kannatuksille. 
 Oliko heidän todellinen kannatus luottamusvälillä? Käytä apuna sivua \url{https://fi.wikipedia.org/wiki/Suomen_presidentinvaali_2018}
 Laske hylätty ääni mukaan kokonaisäänimäärään.
\end{exercise}


\begin{exercise}
 Osoitteessa \url{https://www.vauva.fi/keskustelu/3080040/eduskuntavaali-2019-gallup} kysytään, mitä puoluetta 
 korkea\-tasoi\-ses\-ta keskustelusta tunnetun Vauva-lehden lukijat aikovat äänestää eduskuntavaaleissa 2019.
 Klikkaa ``näytä vastaukset'' vaihtoehtojen alapuolella nähdäksesi tulokset.\\
 a) Mikä puolue olisi kyselyn perusteella vaalien jälkeen eduskunnan suurin?\\
 b) Mikä on 95\%:n luottamusväli puolueen kannatukselle?\\
 c) Onko a- ja b-kohdan ennusteet päteviä? Miksi/miksi ei?
\end{exercise}


\section{Lisätehtäviä}

Kaikki nämä ovat vanhoja matematiikan yo-tehtäviä. Tehtävät eivät ole missään erityisessä järjestyksessä.

\begin{exercise}
 (Syksy 17, lyhyt)
 
 \includegraphics[width=10cm]{ex1.png}
\end{exercise}


\begin{exercise}
 (Kevät 16, lyhyt)
 
 \includegraphics[width=10cm]{ex2.png}
\end{exercise}

\begin{exercise}
 (Kevät 16, lyhyt)
 
 \includegraphics[width=10cm]{ex3.png}
\end{exercise}

\begin{exercise}
 (Kevät 11, pitkä. Voi olla vaikea)
 
 \includegraphics[width=10cm]{ex4.png}
\end{exercise}

\begin{exercise}
 (Syksy 10, pitkä)
 
 \includegraphics[width=10cm]{ex5.png}
\end{exercise}


\begin{exercise}
 (Kevät 08, pitkä)
 
 \includegraphics[width=10cm]{ex6.png}
\end{exercise}


\begin{exercise}
 (Kevät 03, pitkä. Ekassa kysymyksessä pitää ajatella ehdollisia todennäköisyyksiä.)
 
 \includegraphics[width=10cm]{ex7.png}
\end{exercise}

\begin{exercise}
 (Syksy 01, pitkä)
 
 \includegraphics[width=10cm]{ex8.png}
\end{exercise}

\begin{exercise}
 (Kevät 01, pitkä)
 
 \includegraphics[width=10cm]{ex9.png}
\end{exercise}

\begin{exercise}
 (Syksy 99, pitkä)
 
 \includegraphics[width=10cm]{ex10.png}
\end{exercise}

\begin{exercise}
 (Syksy 12, lyhyt)
 
 \includegraphics[width=10cm]{ex11.png}
\end{exercise}

\begin{exercise}
 (Kevät 12, lyhyt)
 
 \includegraphics[width=10cm]{ex12.png}
\end{exercise}

\textsc{Odotusarvot}

\begin{exercise}
 Heität noppaa 600 kertaa. Mikä on kutosten lukumäärän odotusarvo?
\end{exercise}

\begin{exercise}
 Pelaat peliä, jossa heitetään noppaa. Voitat parillisilla silmäluvuilla silmäluvun verran euroissa ja parittomilla häviät 
 kolme euroa. Onko peli sinulle kannattava?
\end{exercise}


\begin{exercise}
 Henkilö pelaa Monopolia. Hän heittää kahta noppaa. Silmäluvuilla 7 hän joutuu Mannerheimintielle ja joutuu maksamaan 24 000 pelirahaa. Silmäluvulla
 9 hän joutuu Erottajalle ja maksaa 40 000 pelirahaa. Silmäluvulla 10 ja suuremmilla hän ylittää lähtöruudun ja saa 4000 pelirahaa. Muussa tapauksessa
 hän ei saa eikä menetä rahaa. Mikä on rahan saamisen
 odotusarvo?
 
\end{exercise}
\end{document}

